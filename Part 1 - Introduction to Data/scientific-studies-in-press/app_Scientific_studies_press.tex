\documentclass[12pt]{article}
%%%%%%%%%%%%%%%%
% Packages
%%%%%%%%%%%%%%%%

\usepackage[top=2cm,bottom=2cm,left=2cm,right= 2cm]{geometry}
\usepackage[parfill]{parskip}
\usepackage{graphicx, fontspec, xcolor,multicol, enumitem, setspace, amsmath, changepage}
\DeclareGraphicsRule{.tif}{png}{.png}{`convert #1 `dirname #1`/`basename #1 .tif`.png}

%%%%%%%%%%%%%%%%
% User defined colors
%%%%%%%%%%%%%%%%

% Pantone 2015 Fall colors
% http://iwork3.us/2015/02/18/pantone-2015-fall-fashion-report/
% update each semester or year

\xdefinecolor{custom_blue}{rgb}{0, 0.32, 0.48} % FROM SPRING 2016 COLOR PREVIEW
\xdefinecolor{custom_darkBlue}{rgb}{0.20, 0.20, 0.39} % Reflecting Pond  
\xdefinecolor{custom_orange}{rgb}{0.96, 0.57, 0.42} % Cadmium Orange
\xdefinecolor{custom_green}{rgb}{0, 0.47, 0.52} % Biscay Bay
\xdefinecolor{custom_red}{rgb}{0.58, 0.32, 0.32} % Marsala

\xdefinecolor{custom_lightGray}{rgb}{0.78, 0.80, 0.80} % Glacier Gray
\xdefinecolor{custom_darkGray}{rgb}{0.35, 0.39, 0.43} % Stormy Weather

%%%%%%%%%%%%%%%%
% Color text commands
%%%%%%%%%%%%%%%%

%orange
\newcommand{\orange}[1]{\textit{\textcolor{custom_orange}{#1}}}

% yellow
\newcommand{\yellow}[1]{\textit{\textcolor{yellow}{#1}}}

% blue
\newcommand{\blue}[1]{\textit{\textcolor{blue}{#1}}}

% green
\newcommand{\green}[1]{\textit{\textcolor{custom_green}{#1}}}

% red
\newcommand{\red}[1]{\textit{\textcolor{custom_red}{#1}}}

%%%%%%%%%%%%%%%%
% Coloring titles, links, etc.
%%%%%%%%%%%%%%%%

\usepackage{titlesec}
\titleformat{\section}
{\color{custom_blue}\normalfont\Large\bfseries}
{\color{custom_blue}\thesection}{1em}{}
\titleformat{\subsection}
{\color{custom_blue}\normalfont}
{\color{custom_blue}\thesubsection}{1em}{}

\newcommand{\ttl}[1]{ \textsc{{\LARGE \textbf{{\color{custom_blue} #1} } }}}

\newcommand{\tl}[1]{ \textsc{{\large \textbf{{\color{custom_blue} #1} } }}}

\usepackage[colorlinks=false,pdfborder={0 0 0},urlcolor= custom_orange,colorlinks=true,linkcolor= custom_orange, citecolor= custom_orange,backref=true]{hyperref}

%%%%%%%%%%%%%%%%
% Instructions box
%%%%%%%%%%%%%%%%

\newcommand{\inst}[1]{
\colorbox{custom_blue!20!white!50}{\parbox{\textwidth}{
	\vskip10pt
	\leftskip10pt \rightskip10pt
	#1
	\vskip10pt
}}
\vskip10pt
}

%%%%%%%%%%%
% App Ex number    %
%%%%%%%%%%%

% DON'T FORGET TO UPDATE

\newcommand{\appno}[1]
{1}

%%%%%%%%%%%%%%
% Turn on/off solutions       %
%%%%%%%%%%%%%%

% Off
\newcommand{\soln}[2]{$\:$\\ \vspace{#1}}{}

%% On
%\newcommand{\soln}[2]{\textit{\textcolor{custom_red}{#2}}}{}

%%%%%%%%%%%%%%%%
% Document
%%%%%%%%%%%%%%%%

\begin{document}
\fontspec[Ligatures=TeX]{Helvetica Neue Light}

Dr. \c{C}etinkaya-Rundel
Duke University - Department of Statistical Science \hfill \\

\ttl{Application exercise \appno{}: \\
Scientific studies in the press}

Below are two studies published in peer reviewed journals, along with their media coverage.
You are asked to question a series of questions about each.

It is possible that the media article doesn't provide the relevant information to answer certain 
question(s). If so, refer to the original study. You should not need to read the entire paper. Simply 
find the information that will help you answer the questions. \\

\section{Haters gonna hate, study confirms}

\begin{minipage}[t]{0.49\textwidth}
\textbf{Media coverage:} \\
{\small
Haters Are Gonna Hate, Study Confirms \\
By  Katy Waldman \\
Published: August 28, 2013 on Slate \\
Link: \url{http://slate.me/1dr5XPE} \\
}
\end{minipage}
\begin{minipage}[t]{0.02\textwidth}
$\:$ \\
\end{minipage}
\begin{minipage}[t]{0.49\textwidth}
\textbf{Original study:} \\
{\small
Hepler, J., \& Albarracin, D. (2013). ``Attitudes without objects: Evidence for a 
dispositional attitude, its measurement, and its consequences". \textit{Journal of 
personality and social psychology}, 104(6), 1060. \\
Link: \url{https://sakai.duke.edu/x/oIukKJ} \\
}
\end{minipage}

\section{Rich more likely to take candy from babies}

* Focus only on the candy study.

\begin{minipage}[t]{0.49\textwidth}
\textbf{Media coverage:} \\
{\small
Study: Rich more likely to take candy from babies \\
By Ezra Klein \\
Published: February 27, 2012 on WaPo \\
Link: \url{http://wapo.st/Am2OBk} \\
}
\end{minipage}
\begin{minipage}[t]{0.02\textwidth}
$\:$ \\
\end{minipage}
\begin{minipage}[t]{0.49\textwidth}
\textbf{Original study:} \\
{\small
Piff, Paul K., et al. "Higher social class predicts increased unethical behavior." 
\textit{Proceedings of the National Academy of Sciences}, 109.11 (2012): 4086-4091.
Link: \url{http://www.pnas.org/content/109/11/4086.full.pdf} \\
}
\end{minipage}

\pagebreak

\section{Questions}

\begin{enumerate}

\item What are the cases? \\

%Haters: 200 men and women  \\
%Candy: 129 undergraduates

\item What is (are) the response variable(s) in this study? \\

%Haters: Attitude towards the microwave oven \\
%Candy: Number of pieces of candy participants took

\item What is (are) the explanatory variable(s) in this study? \\

%Haters: Whether the participant is a hater or not \\
%Candy: Whether participant views him/herself as wealthy or poor

\item Does the study employ random sampling? How about random assignment? \\

%Haters: Via Amazon's MTurk - self selected to be an MTurk, but randomly sampled for study, no random assignment. \\
%Candy: We don't know if the sample is random, but there is no random assignment.

\item Is this an observational study or an experiment? Explain your reasoning. \\

%Haters: Observational, doesn't use random assignment. \\
%Candy: Observational, doesn't use random assignment.

\item Can we establish a causal link between the explanatory and response variables? \\

%Haters: No \\
%Candy: No

\item Can the results of the study be generalized to the population at large? \\

%Haters: No \\
%Candy: Yes, if MTurk sample is representative

\end{enumerate}


\end{document}